\begin{abstract}
\title{ADSORBATE INDUCED RECONSTRUCTIONS ON METAL SURFACES}
\author{Joseph R. Michalka}
In this dissertation I present work on the modeling of adsorbate-metal
interactions with a specific focus on Carbon Monoxide (CO) induced
restructuring of Platinum stepped surfaces. Additionally, I will also present
work on the development of a multiple-minima fluctuating charge (MM-flucq)
electrostatic interaction potential which has been use to simulate charge
transfer events and oxide formation on Platinum surfaces.

New Pt-CO and Au-CO forcefields were developed to study coverage-dependent
CO-induced restructuring of high-index Platinum \& Gold (557) surfaces. It was observed
that weak Au-CO binding led to minimal disruption of the surface, whereas the
strong Pt-CO binding interactions resulted in significant disruption of the
step-edges leading to step-wandering on the surface. The highest coverage (0.5
ML) Pt-CO systems also underwent step-doubling which had been seen in an
earlier experiment. Our proposed mechanism for this reconstruction is based on
the strong CO-CO quadrupolar repulsion that causes an increase of adatom surface
mobility. The increased mobility directly leads to greater stochastic
step-wandering, which in turn eventually led to double-layer formation.

A Platinum/Palladium (557) surface was simulated to explore the interplay between
CO-induced reconstruction on a more complicated bimetallic surface. The Pt-CO
forcefield was retuned while a new Pd-CO forcefield was developed. A pure Pd
(557) system was also explored as a control case. The difference in binding
strengths was found to play an important role in the disruption of the Pt/Pd
surface while the preferred binding sites on each system (Pt: atop, Pd: bridge/hollow)
led to significantly different behaviors with regards to surface diffusion and
mobility.

The observation from earlier work of the importance of step-edges energetics
when examining adatom formation encouraged us to directly examine the effect of
straight edges (557) \& (112) and kinked edges (765) \& (321), along with the effect of
plateau lengths on the CO-induced reconstruction of Pt surfaces. As
hypothesized, the systems with rougher edges experienced a greater amount of
surface diffusion and a concomitantly increased amount of step-wandering. The
length of the (111) plateaus between step edges also played an important role
with regard to the time taken for the systems to undergo step doubling.

Accurate treatment of the electrostatic interactions between adsorbates and
surfaces is often neglected in molecular dynamics simulations because of the
increased computational cost and difficulty of implementation. Instead, the
attractive or repulsive effects of the charge interactions are partioned
amongst the other interatomic potentials terms during fitting. Our new MM-flucq
potential allows us to more properly describe a metal surface's response to
impinging charged species in a dynamic fashion by allowing the charges on each
atomic site to flucuate in response to the local electronic environment.

\end{abstract}

