\begin{abstract}
\label{chap:abstract}
\title{ADSORBATE INDUCED RECONSTRUCTIONS ON METAL SURFACES}
\author{Joseph R. Michalka}
In this dissertation I present work on the modeling of adsorbate-metal
interactions with a specific focus on Carbon Monoxide (\ce{CO}) induced
restructuring of Platinum stepped surfaces. Additionally, I will present work
on the development of a multiple-minima fluctuating charge (MM-flucQ) potential
which has been used to simulate charge responsive Platinum surfaces.

New Pt-CO and Au-CO forcefields were developed to study coverage-dependent
restructuring of high-index Platinum \& Gold (557) surfaces. It was observed
that the weak Au-CO binding led to minimal disruption of the surface, whereas
the strong Pt-CO interactions resulted in significant disruption of the
step-edges which led to increased step-wandering. Specifically, the strong
CO-CO quadrupolar repulsion caused an increase in adatom mobility. This
increased mobility eventually led to large-scale surface reconstructions
including the formation of a metastable double layer.

A Platinum/Palladium (\ce{Pt}/\ce{Pd}) (557) surface was simulated to explore
CO-induced reconstructions on a more complicated bimetallic surface. The Pt-CO
forcefield was retuned while a new Pd-CO forcefield was developed.  The
difference in binding strengths was found to play an important role in the
disruption of the Pt/Pd surface while the preferred binding sites on each
system (Pt: atop, Pd: bridge/hollow) led to significantly different behaviors
with regards to surface diffusion and mobility.

The importance of step-edge energetics for the formation of adatoms encouraged
us to directly examine straight edged (557) \& (112) and kinked edged (765) \&
(321) Platinum surfaces.  As hypothesized, the systems with rougher edges
experienced a greater amount of surface diffusion and a concomitantly increased
amount of step-wandering. The length of the (111) plateaus between step edges
also played an important role with regard to the extent of surface
reconstruction observed.

Accurate treatment of the electrostatic interactions between adsorbates and
surfaces is often neglected in molecular dynamics simulations because of the
increased computational cost and difficulty of implementation.  Our new
MM-flucQ potential allows us to more properly describe a metal surface's
response to impinging charged species in a dynamic fashion by allowing the
charges on each atomic site to fluctuate in response to the local electronic
environment.
\end{abstract}

