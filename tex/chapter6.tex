\chapter{CONCLUSIONS}

%

In this dissertation I have presented new forcefields for \ce{CO} interacting
with \ce{Pt} as well as illustrating the mechanisms of surface reconstructions
as it relates to step-edge doubling and island formation. I have further examined the effects of displayed step-type on the 


\subsection{Alternative Systems of Interest}
This dissertation has focused on the well-studied \ce{Pt}-\ce{CO} interactions
and only branched out by examining \ce{CO} interactions with \ce{Au} and
\ce{Pd}.  There are numerous adsorbate-metal systems that have been identified
as also undergoing various types of reconstruction including \ce{CO} and
\ce{O2} over \ce{Rh\bond{-}Pt} core-shell nanoparticles,\citep{Tao:2008aa},
\ce{CO} over \ce{Cu} (111),\citep{Eren:2016qt} \ce{CO} and \ce{O2} over a (553)
\ce{Rh} surface,\citep{Zhang:2015zr}, \ce{CO} over a \ce{Pd/Au} bimetallic
surface,\citep{Kim:2013mi} and many more. The majority of these are experimental
studies which are often unable to directly identify the mechanism of the
reconstruction making them of potential interest to model.

Despite \ce{Pt\bond{-}Pd} bimetallic catalysts being of considerable interest,
both components are relatively expensive and significant effort is being
expended to tailor \ce{Pt\bond{-}M} surfaces and nanoparticles.
\ce{Pt\bond{-}Ni} is of immense interest because of its increased activity for
the oxygen reduction reaction and would be of much interest to
model.\citep{Tuaev:2013fk, Stamenkovic:2007kk,Sneed:2014fj}

As the Multiple Minima Fluctuating Charge method is continually refined, many
avenues for exploration are opened, especially those involving charged
adsorbates on metal surfaces. Many systems and processes of interest, including
\ce{H2O} on \ce{Pt},\citep{Xu:0dz} metallic-oxide
formation,\citep{Streitz:1994mw, Fantauzzi:2014pb, Lloyd:2016jt}, and various
biological molecules on metal surfaces,\citep{Padmos:0qf, Mete:2015rc} which
have been studied at differing levels of theory would be benefitted by more
fully treating the electrostatic interactions present in the system. 
