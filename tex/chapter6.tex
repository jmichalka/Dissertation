\chapter{CONCLUSIONS}


The surfaces of metals are a complex landscape and their study is of great
importance for catalyst design. In this dissertation, the specifics of \ce{CO}
induced reconstructions on \ce{Pt} was explored and a mechanism for the
experimentally observed doubling was proposed. A more complicated bimetallic
\ce{Pt\bond{-}Pd} near-surface-alloy was also modeled to determine the effects
of different binding strengths on the stability of the outer \ce{Pt} layer.
Carbon monoxide's interaction with \ce{Pt} was revisited by examining how the
presence of different surface steps would affect the previously observed
reconstruction. Finally, a new method for treating fluctuating charges
(MM-flucQ) was proposed that would allow for metal surfaces to be charge
responsive to impingent adsorbates.

In studying the induced reconstruction of \ce{Pt}, the \ce{Pt\bond{-}CO}
interaction was newly parameterized to better capture the preferred atop
binding site. The effects of coverage were also examined by exposing (557)
\ce{Pt} surfaces to five different CO coverages [0, 0.05, 0.25, 0.33, 0.50].
Additionally, the importance of the identity of the metal was tested by
examining a similar system made of \ce{Au}. It was observed that the strong
quadrupolar moment of \ce{CO}, along with its preferred atop binding site, was
integral to the initial breaking of the clean (557) surface. The strong binding
of \ce{Pt} to \ce{CO}, coupled with the large \ce{CO\bond{-}CO} repulsion led
to an increase in \ce{Pt} adatom diffusion on the surface. The adatoms were
observed to explore significant amounts of the surface as well as move up and
down step-edges. These results allowed us to propose our mechanism that high
coverages of \ce{CO} help break up the step-edges after which double layer
formation results from stochastic step-wandering.




\subsection{Alternative Systems of Interest}
This dissertation has focused on the well-studied \ce{Pt}-\ce{CO} interactions
and only branched out by examining \ce{CO} interactions with \ce{Au} and
\ce{Pd}.  There are numerous adsorbate-metal systems that have been identified
as also undergoing various types of reconstruction including \ce{CO} and
\ce{O2} over \ce{Rh\bond{-}Pt} core-shell nanoparticles,\citep{Tao:2008aa},
\ce{CO} over \ce{Cu} (111),\citep{Eren:2016qt} \ce{CO} and \ce{O2} over a (553)
\ce{Rh} surface,\citep{Zhang:2015zr}, \ce{CO} over a \ce{Pd/Au} bimetallic
surface,\citep{Kim:2013mi} and many more. The majority of these are experimental
studies which are often unable to directly identify the mechanism of the
reconstruction making them of potential interest to model.

Despite \ce{Pt\bond{-}Pd} bimetallic catalysts being of considerable interest,
both components are relatively expensive and significant effort is being
expended to tailor \ce{Pt\bond{-}M} surfaces and nanoparticles.
\ce{Pt\bond{-}Ni} is of immense interest because of its increased activity for
the oxygen reduction reaction and would be of much interest to
model.\citep{Tuaev:2013fk, Stamenkovic:2007kk,Sneed:2014fj}

As the Multiple Minima Fluctuating Charge method is continually refined, many
avenues for exploration are opened, especially those involving charged
adsorbates on metal surfaces. Many systems and processes of interest, including
\ce{H2O} on \ce{Pt},\citep{Xu:0dz} metallic-oxide
formation,\citep{Streitz:1994mw, Fantauzzi:2014pb, Lloyd:2016jt}, and various
biological molecules on metal surfaces,\citep{Padmos:0qf, Mete:2015rc} which
have been studied at differing levels of theory would be benefitted by more
fully treating the electrostatic interactions present in the system. 
