\chapter{SUMMARY}
\label{chap:summary}

The surfaces of metals are a complex landscape and their study is of great
importance for catalyst design. In this dissertation, the specifics of \ce{CO}
induced reconstructions on \ce{Pt} was explored and a mechanism for the
experimentally observed step-edge doubling was proposed. A more complicated
bimetallic \ce{Pt\bond{-}Pd} near-surface-alloy was also modeled to determine
the effects of different binding strengths on the stability of the outer
\ce{Pt} layer.  Carbon monoxide's interaction with \ce{Pt} was further
elucidated by examining how the presence of different surface steps would
affect the previously observed reconstruction. Finally, a new method for
treating fluctuating charges (mm-FlucQ) was proposed that would allow for metal
surfaces to dynamically respond to impinging charged adsorbates.

In studying the induced reconstruction of \ce{Pt}, the \ce{Pt\bond{-}CO}
interaction was newly parameterized to better capture the preferred atop
binding site. The effects of coverage were also examined by exposing (557)
\ce{Pt} surfaces to five different CO coverages (0 ML, 0.05 ML, 0.25 ML, 0.33
ML, 0.50 ML).  Additionally, the importance of the identity of the metal was
tested by examining a similar system made of \ce{Au}. Throughout the
simulations it was observed that the strong quadrupolar moment of \ce{CO},
along with its preferred atop binding site, was integral to the initial
breaking of the clean (557) surface. The strong binding of \ce{CO} to \ce{Pt},
coupled with the large \ce{CO\bond{-}CO} repulsion led to an increase in
\ce{Pt} adatom diffusion on the surface. The adatoms were observed to explore
significant amounts of the surface as well as traverse up and down step-edges.
These results allowed us to propose a mechanism that depends on high coverages
of \ce{CO} helping to break up step-edges after which double layer formation
results from stochastic step-wandering ``zippering''.

Research into bimetallic systems and their high tunability led us to model a
\ce{Pt\bond{-}Pd} near-surface-alloy (NSA) again under a \ce{CO} atmosphere. A
pure \ce{Pd} system was also examined as a control case, where no
reconstruction was observed over the simulation time. The bimetallic system
however showed a significant amount of deviation from the initial (557) surface
and the differing binding strengths between \ce{CO} and \ce{Pd} vs. \ce{CO} and
\ce{Pt} played a central role in this process. The preferred bridge binding for
\ce{Pd\bond{-}CO} also had the effect of lowering the \ce{Pd} system's surface energy making it
more stable, whereas the atop binding preference on \ce{Pt} had the effect of
raising the surface energy making adatom formation and the concomitant increase
in diffusion more favored. As observed in Figure \ref{fig:nearestNeighbors},
the movement of \ce{Pt} led to an increase in the number of
\ce{Pt\bond{-}Pt} bonds which resulted in island formation on the surface of
the underlying \ce{Pd}. The surface domain analysis, implemented for this
study, was useful in measuring the change in domain size over time and as a
function of \ce{CO} coverage which helped us conclude that the sintering we
observed was primarily a result of a a kinetic barrier that needed to be
overcome, but that the presence of \ce{CO} sped up the process and allowed it
to reach a separate equilibrium as compared to the system run without \ce{CO}.

A further investigation into \ce{CO}'s effect on \ce{Pt} was carried out to
better elucidate if the (557) structure of \ce{Pt} was unique in its
restructuring or if other high-index surfaces would undergo similar
reconstructions. The lengths of the (111) plateaus along with the type of step
edge, straight or kinked, were both explored through the four systems that were
examined [(112), (321), (557), (765)]. While significant adatom movement and
step-edge wandering was observed in the majority of the systems, only the (321)
system showed the expected doubling during our simulations. Most surprising
was that the (557) system did not undergo doubling in the simulation time
suggesting that either the stochastic nature of the process was not adequately
sampled during the finite simulation or that the modification to the
\ce{Pt\bond{-}CO} forcefield to reflect recent
experiments\citep{Deshlahra:2012aa} was sufficient to modify the potential
energy landscape away from a stable double step.

% Section on MM-flucq, but don't have many results yet so...
%
%Accurately describing the interaction between adsorbates and metal surfaces led
%us to rethink the current approaches for describing those interactions. The
%majority of molecular dynamics simulations that include large metal clusters or
%surfaces often ignore the electronic nature of the metal. 
The development of the multiple-minima fluctuating charge model has allowed for
the incorporation of charge transfer within a metal surface and in response to
local electric fields. Parameterization is on-going but preliminary results
have allowed the modeling of a \ce{Pt} surface that is charge responsive. The
surface is able to develop an image charge in response to an impinging \ce{O-}.
The accumulation of positive charge is temporary and returns to equilibrium
when the \ce{O-} bounces away from the surface. While the fitting of these
interactions is non-trivial, this method will open up a new class of systems
for accurate modeling in molecular dynamics.

%Conclusion of my work
As the need for clean sources of energy and efficient synthesis of various
products continues to grow, so to will catalyst design continue to increase in
importance.  Improvements in experimental and theoretical techniques have led
to the creation of single-atom catalysts,\citep{Qiao:2011zp, Yang:2013sf}, the
bridging of the pressure and temperature gaps,\citep{Tao:2010aa, Eren:2016qt}
as well as elucidating mechanisms of adsorbate induced
reconstructions. \citep{Michalka:2013aa,Kim:2016cr} This work has contributed to
this endeavor by further characterizing \ce{Pt\bond{-}CO} systems while also
developing new forcefields and approaches to more accurately treat
metal-adsorbate interactions.



\section{Alternative systems of interest}
This dissertation has focused on the well-studied \ce{Pt}-\ce{CO} interactions
and only branched out by examining \ce{CO} interactions with \ce{Au} and
\ce{Pd}.  There are numerous adsorbate-metal systems that have been identified
as also undergoing various types of reconstruction including \ce{CO} and
\ce{O2} over \ce{Rh\bond{-}Pt} core-shell nanoparticles,\citep{Tao:2008aa},
\ce{CO} over \ce{Cu} (111),\citep{Eren:2016qt} \ce{CO} and \ce{O2} over a (553)
\ce{Rh} surface,\citep{Zhang:2015zr}, \ce{CO} over a \ce{Pd/Au} bimetallic
surface,\citep{Kim:2013mi} and many more. The majority of these are experimental
studies which are often unable to directly identify the mechanism of the
reconstruction making them of potential interest to model.

Despite \ce{Pt\bond{-}Pd} bimetallic catalysts being of considerable interest,
both components are relatively expensive and significant effort is being
expended to tailor \ce{Pt\bond{-}M} surfaces and nanoparticles.
\ce{Pt\bond{-}Ni} is of immense interest because of its increased activity for
the oxygen reduction reaction and would be a good candidate for future
simulations.\citep{Sneed:2014fj, Stamenkovic:2007kk, Tuaev:2013fk}

As the multiple minima fluctuating charge method is continually refined, many
avenues for exploration are opened, especially those involving charged
adsorbates on metal surfaces. Many systems and processes of interest, including
\ce{H2O} on \ce{Pt},\citep{Xu:2016dz} metallic-oxide
formation,\citep{Streitz:1994mw, Fantauzzi:2014pb, Lloyd:2016jt}, and various
biological molecules on metal surfaces,\citep{Padmos:0qf, Mete:2015rc} which
have been studied at differing levels of theory would be benefited by more
fully treating the electrostatic interactions present in the system. 
