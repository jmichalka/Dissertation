%%
%% This is file `template.tex',
%% generated with the docstrip utility.
%%
%% The original source files were:
%%
%% nddiss2e.dtx  (with options: `template')
%% 
%% This is a generated file.
%% 
%%  Copyright (C) 2004-2005 Sameer Vijay
%% 
%%  This file may be distributed and/or modified under the
%%  conditions of the LaTeX Project Public License, either
%%  version 1.2 of this license or (at your option) any later
%%  version. The latest version of this license is in
%%     http://www.latex-project.org/lppl.txt
%% 
%% 
%% ==============================================================
%% 
%% Notre Dame's Dissertation document class by Sameer Vijay
%% that adheres to the University of Notre Dame guidelines
%% published in Spring 2004. Updated by Megan Patnott to adhere
%% to the University of Notre Dame guidelines as of Spring 2013.
%% 
%% Please send any improvements/suggestions to :
%%     Shari Hill, Graduate Reviewer.
%%     shill2@nd.edu
%% 
%% For documentation on how to use nddiss2e class, process the
%% file nddiss2e.dtx through LaTeX.
%% 
%% ==============================================================
%% 
\ProvidesFile{template.tex}
    [2013/04/16 v3.2013^^J%
     Template file for NDdiss2e class by Sameer Vijay and updated by Megan Patnott^^J]
\documentclass[draft]{nddiss2e}
                     % One of the options draft, review, final must be chosen.
                     % One of the options textrefs or numrefs should be chosen
                     % to specify if you want numerical or ``author-date''
                     % style citations.
                     % Other available options are:
                     % 10pt/11pt/12pt (available with draft only)
                     % twoadvisors
                     % noinfo (should be used when you compile the final time
                     %         for formal submission)
                     % sort (sorts multiple citations in the order that they're
                     %       listed in the bibliography)
                     % compress (compresses numerical citations, e.g. [1,2,3]
                     %           becomes [1-3]; has no effect when used with
                     %           the textrefs option)
                     % sort&compress (sorts and compresses numerical citations;
                     %           is identical to sort when used with textrefs)

\begin{document}

\frontmatter         % All the items before Chapter 1 go in ``frontmatter''

\title{ADSORBATE INDUCED RECONSTRUCTIONS OF METAL SURFACES}            % TITLE OF WORK. It must be in all caps, and ensuring this is your
 % responsiblity.
\author{Joseph R. Michalka}           % Author's name \work{ }             % ``Dissertation'' or ``Thesis''
\work{Dissertation}
\degaward{Doctor of Philosophy}         % Degree you're aiming for. Should be one of the following options:
 % ``Doctor of Philosophy'' (do NOT include ``in Subject'')
 % ``Master of Science // in // Subject''
\advisor{J. Daniel Gezelter}          % Advisor's name
 % \secondadvisor{ } % Second advisor, if used option ``twoadvisors''
\department{Chemistry and Biochemistry}       % Name of the department

\maketitle           % The title page is created now

 % You must use either the \makecopyright option or the \makepublicdomain option.
 % \copyrightholder{ } % If you're not the copyright holder
 % \copyrightyear{ }   % If the copyright is not for the current year
 % \makecopyright      % If not making your work public domain
                       % uncomment out \makecopyright
 % \makepublicdomain   % Uncomment this to make your work public domain

 % Including an abstract is optional for a master's thesis, and required for a
 % doctoral dissertation.
\begin{abstract}
\title{ADSORBATE INDUCED RECONSTRUCTIONS ON METAL SURFACES}
\author{Joseph R. Michalka}
In this dissertation I present work on the modeling of adsorbate-metal
interactions with a specific focus on Carbon Monoxide (CO) induced
restructuring of Platinum stepped surfaces. Additionally, I will also present
work on the development of a multiple-minima fluctuating charge (MM-flucq)
electrostatic interaction potential which has been use to simulate charge
transfer events and oxide formation on Platinum surfaces.

New Pt-CO and Au-CO forcefields were developed to study coverage-dependent
CO-induced restructuring of high-index Platinum \& Gold (557) surfaces. It was observed
that weak Au-CO binding led to minimal disruption of the surface, whereas the
strong Pt-CO binding interactions resulted in significant disruption of the
step-edges leading to step-wandering on the surface. The highest coverage (0.5
ML) Pt-CO systems also underwent step-doubling which had been seen in an
earlier experiment. Our proposed mechanism for this reconstruction is based on
the strong CO-CO quadrupolar repulsion that causes an increase of adatom surface
mobility. The increased mobility directly leads to greater stochastic
step-wandering, which in turn eventually led to double-layer formation.

A Platinum/Palladium (557) surface was simulated to explore the interplay between
CO-induced reconstruction on a more complicated bimetallic surface. The Pt-CO
forcefield was retuned while a new Pd-CO forcefield was developed. A pure Pd
(557) system was also explored as a control case. The difference in binding
strengths was found to play an important role in the disruption of the Pt/Pd
surface while the preferred binding sites on each system (Pt: atop, Pd: bridge)
led to significantly different behaviors with regards to surface diffusion and
mobility.

The observation from earlier work of the importance of step-edge energetics
encouraged us to directly examine the effect of straight edges (557) \& (112)
and kinked edges (321) \& (765), along with plateau lengths on the CO-induced
reconstruction of Pt surfaces. As hypothesized, the systems with rougher edges
experienced a greater amount of surface diffusion and a concomitantly increased
amount of step-wandering. The length of the (111) plateaus between step edges
also played an important role with regard to the time taken for the systems to
undergo step doubling.

Accurate treatment of the electrostatic interactions between adsorbates and
surfaces is often neglected in molecular dynamics simulations because of the
increased computational cost and difficulty of implementation. Instead, the
attractive or repulsive effects of the charge interactions are partioned
amongst the other interatomic potentials terms during fitting. Our new MM-flucq
potential allows us to more properly describe a metal surface's response to
impinging charged species in a dynamic fashion.

\end{abstract}
 %                         % Either place the text between begin/end, or
 % \begin{abstract}
\label{chap:abstract}
\title{ADSORBATE INDUCED RECONSTRUCTIONS ON METAL SURFACES}
\author{Joseph R. Michalka}
In this dissertation I present work on the modeling of adsorbate-metal
interactions with a specific focus on carbon monoxide (\ce{CO}) induced
restructuring of platinum stepped surfaces. Additionally, I will present work
on the development of a multiple-minima fluctuating charge (mm-FlucQ) potential
which has been used to simulate charge responsive platinum surfaces.

New Pt-CO and Au-CO forcefields were developed to study coverage-dependent
restructuring of high-index platinum \& gold (557) surfaces. It was observed
that the weak Au-CO binding led to minimal disruption of the surface, whereas
the strong Pt-CO interactions resulted in significant disruption of the
step-edges which led to increased step-wandering. Specifically, the strong
CO-CO quadrupolar repulsion caused an increase in adatom mobility. This
increased mobility eventually led to large-scale surface reconstructions
including the formation of a metastable double layer.

A platinum/palladium (\ce{Pt}/\ce{Pd}) (557) surface was simulated to explore
CO-induced reconstructions on a more complicated bimetallic surface. The Pt-CO
forcefield was retuned while a new Pd-CO forcefield was developed.  The
difference in binding strengths was found to play an important role in the
disruption of the Pt/Pd surface while the preferred binding sites on each
system (Pt: atop, Pd: bridge/hollow) led to significantly different behaviors
with regards to surface diffusion and mobility.

The importance of step-edge energetics for the formation of adatoms encouraged
us to directly examine straight edged (557) \& (112) and kinked edged (765) \&
(321) platinum surfaces.  As hypothesized, the systems with rougher edges
experienced a greater amount of surface diffusion and a concomitantly increased
amount of step-wandering. The length of the (111) plateaus between step edges
also played an important role with regard to the extent of surface
reconstruction observed.

Accurate treatment of the electrostatic interactions between adsorbates and
surfaces is often neglected in molecular dynamics simulations because of the
increased computational cost and difficulty of implementation.  Our new
mm-FlucQ potential allows us to more properly describe a metal surface's
response to impinging charged species in a dynamic fashion by allowing the
charges on each atomic site to fluctuate in response to the local electronic
environment.
\end{abstract}

  % put it in a file to be included

 % Including a dedication is optional.
%\renewcommand{\dedicationname}{\mbox{}} % Replace \mbox{} if you want
\renewcommand{\dedicationname}{DEDICATION} % Replace \mbox{} if you want
                                           % something else. It must be in
                                           % all caps, and doing so is your
                                           % responsibility.
\begin{dedication}
To my wife and family, and the many great teachers who have always believed in me.
\end{dedication}
 %                       % Use one of the two choices to add dedication text
 % \include{dedication}

\tableofcontents
\listoffigures
\listoftables

 % Including a list of symbols is optional.
 %% \renewcommand{\symbolsname}{newsymname} % Replace ``newsymname'' with
                                            % the name you want, and uncomment
                                            % The name must be in all caps,
                                            % and ensuring this is your
                                            % responsibility
 % \begin{symbols}
 % \end{symbols}
 %                       % Use one of the two choices to add symbols text
 % \include{symbols}

 % Including a preface is optional.
 %% \renewcommand{\prefacename}{ } % If you want another Preface name, add
                                   % something else, and uncomment.
                                   % The name must be in all caps, and
                                   % ensuring this is your responsibility.
 % \begin{preface}
 % \end{preface}
 %                       % Use one of the two choices to add preface text
 % \include{preface}

 % Including an acknowledgements section may or may not be optional. It's hard to
 % tell from the information available in Spring 2013.
 %% \renewcommand{\acknowledgename}{ } % If you want another Acknowledgement name
                                       % add something else, and uncomment
                                       % The name must be in all caps, and
                                       % ensuring this is your responsiblity.
\begin{acknowledge}
I would like to thank my advisor, Professor J. Daniel Gezelter, for his
guidance and support throughout this research.
I am indebted to my former and present colleagues, Dr. Shenyu Kuang, Dr. James
Marr, Dr. Kelsey Stocker, Dr. Zack Terranova, Dr. Mary Sherman, Madan
Lamichhane, Patrick Louden, Suzanne Neidhart, Patrick McIntyre, Heather
Chiarello (?), Andrew Latham, and Thomas Parsons for helpful discussions. 
\end{acknowledge}
 %                       % Use one of the two choices to add acknowledge text
 % \begin{acknowledge} 
My time at graduate school has been a long road that has been greatly enriched
by my many fantastic colleagues and mentors. I would like to especially thank
my advisor, Professor J. Daniel Gezelter, for his guidance and support
throughout this research. His piercing insights, expertise in coding, and
skills as an educator have helped me grow as a researcher and a teacher. I am
indebted to my former and present group members and colleagues, Dr. Shenyu
Kuang, Dr. James Marr, Dr.  Kelsey Stocker, Dr. Zack Terranova, Dr. Mary
Sherman, Madan Lamichhane, Patrick Louden, Suzanne Neidhart, Patrick McIntyre,
Heather Chiarello, Andrew Latham, and Thomas Parsons for many helpful
discussions, research related or otherwise.

I am extremely grateful to the Kaneb Center and its staff, especially Kristi
Rudenga, as well as my co-graduate student associates, Andre Audette, Amy
Buchmann, Justus Ghormley, Megan Hall, and Kelly Warmuth (Kuznicki) for many
great discussions and workshops filled with teaching and pedagogy. I also owe
many thanks to Dr. Lappin, James Johnson, Sarah West, Steven Weitstock, and Dr.
Gezelter for the experience I gained while assisting them in classes and labs.

I am very thankful to my committee members, Professor J. Daniel Gezelter,
Professor Steven A. Corcelli, and Professor Gregory V. Hartland for their many
helpful questions and suggestions which have help me navigate through this process. 

I must not forgot to mention the administrative assistance and encouragement I
received from Deb Bennet, Cheryl Copley, Amanda Huerta, and the rest of the
Chemistry and Biochemistry Department.

Finally, I would like to acknowledge my best friend and wife, Karen Hooge
Michalka. The love and support she has shown me throughout my time at Notre
Dame, especially while I have been writing and applying for jobs has helped
more than she knows.
\end{acknowledge}



\mainmatter
 % Place the text body here.
 % \include{chapter-one}
 % Begin each chapter with \chapter{TITLE}. Chapter titles must be in all caps
 % and ensuring that they are is your responsibility.

\chapter{INTRODUCTION}
%Metal surfaces
%  motivation (catalysis, energy generation, biomedical applications)
%  electronic, optical properties (stable at extreme temperatures, pressures for the most part)
%  list off neat things metal surfaces have done (20 or so spread throughout first couple of paragraphs
% Tie into importance of adsorbate interactions
Metal surfaces play a role in many areas of chemistry, including catalysis,
energy generation, and biomedical applications. Their displayed facets and
morphologies often play a significant role in their activity, selectivity, and
stability. In most practical applications the surface of the metal will be
exposed to a variety of atmospheric compositions, typically at high
temperatures and pressures, leading to adsorbed species on the surface. 

\section{Adsorbate Interactions on Metal Surfaces}
%Image of particles adsorbed on the surface
%covalent vs charge-interaction vs attractive by not strongly bound
%binding affected by surface facet (electronic, d-band) and generalized coordinate number
% sabatier/volcano plots
%experimental abilities
%dft capabilities
%where molecular dynamics fits
The frontier orbitals on many common small-molecule adsorbates (CO,
NO\textsubscript{x}, O\textsubscript{2}) tend to match up well with the d-band
center of a significant number of relevant metals leading to extremely strong
binding interactions. These binding interactions are often mediated by the
coverage and specific surface facet displayed.


\subsection{Adsorbate Induced Reconstructions}
% Studies done on clean metal surfaces tend to suffer from pressure and temperature gaps
% high pressure xps, stm, spin echo helium bombardment provide some ways to mimic the environment of industrial catalysts
% Difficult to determine mechanisms because of time and spacial resolution
% DFT good at calculating relative energies for small systems, but the size of these reconstructions makes calculations expensive
% mol dyn. allows us to explore the interactions that lead to adsorbate-induced reconstructions

%\section{Modeling} I think spreading this amongst the earlier sections is better

\chapter{METHODOLOGIES}

\section{Metal Interactions}
% Talk about challenges of metal interactions as compared to simple pair-wise interactions
% energy dependent on local environment
% Introduce EAM, many equations, one to two figures?, justify choice
% Good at mixing, Pt/Pd bimetallic etc.

\section{Metal-Adsorbate Interactions}
% Covalent, non covalent (strength of interaction)
% focus on M-CO, describe M-C, M-O, why repulsive morse
% image of different pieces of the potential from midwest
%

\section{Adsorbate-Adsorbate}

\section{Electrostatics}


\subsection{Multiple Minima Fluctuating Charges (MM-FlucQ)}

\chapter{MOLECULAR DYNAMICS SIMULATIONS OF THE SURFACE RECONSTRUCTIONS OF PT(557) AND AU(557) UNDER EXPOSURE TO CO}


\chapter{CO-INDUCED ISLAND FORMATION ON PT/PD(557) SURFACE ALLOYS: A MOLECULAR DYNAMICS STUDY}


\appendix

 % If you have appendices, add them here.
 % Begin each one with \chapter{TITLE} as before- the \appendix command takes
 % care of renaming chapter headings and creates a new page in the Table of
 % Contents for them.
 % \include{appendix-one}

\backmatter              % Place for bibliography and index


\bibliographystyle{nddiss2e}
 \bibliography{ }           % input the bib-database file name


\end{document}

%%
\endinput
%%
%% End of file `template.tex'.
