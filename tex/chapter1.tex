%Introduction
%  Adsorbate-Metal systems 
%      catalysts
%      devices
%      oxide formation
%    This dissertation
%      structure
%      dynamics
%      reconstruction
%      accurate treatment of electronic interactions and charge transfer effects
%    Organized
%      1: Intro
%      2: Methodology Development
%      3: Pt-CO 557
%      4: Pt/Pd-CO 557
%      5: 112/557/321/765
%      6: MM-FlucQ
%      7: Summary
%
%  Metal Systems:
%      Structure
%        bimetallic (alloys, layers, near surface alloys, etc.)
%      Dynamics
%    Adsorbate Interactions:
%      binding sites
%      patterning
%    Dynamics
%      diffusion
%      step-wandering
%    Reconstruction
%      refaceting
%      doubling
%      island formation

\chapter{INTRODUCTION}
%CHAP1
%Metal surfaces
%  motivation (catalysis, energy generation, biomedical applications)
%  electronic, optical properties (stable at extreme temperatures, pressures for the most part)
%  list off neat things metal surfaces have done (20 or so spread throughout first couple of paragraphs
% Tie into importance of adsorbate interactions
Metal \citep{} surfaces and nanoparticles play a role in many areas of chemistry,
including catalysis, energy generation, and biomedical applicationsTheir
displayed facets and morphologies often play a significant role in their
activity, selectivity, and stability. Bimetallic species and alloys, along with
supported surfaces and nanoparticles allow for an even greater design space for
various mechanical, optical, and catalytical properties. In most practical
applications the surface of the metal will be exposed to a variety of
atmospheric compositions leading to adsorbed species on the surface and
potentially oxide formation. The interactions between metals and adsorbates
adds another layer of complexity when attempting to fully describe metal
surfaces.

In general, the structure of metal surfaces can be strongly perturbed by the
presence of adsorbates. The extent of coverage and adsorbate self-interactions
can both drastically affect the low-energy surface, sometimes leading to
surface reconstruction. Additionally, since the efficacy of metal surfaces
typically strongly depends on the exposed structure, having a detailed
understanding of the mechanism of surface dynamics and restructuring as caused
by adsorbates is of the utmost relevance. This dissertation is intended to
provide a fundamental picture for adsorbate induced reconstruction on metal
surfaces by examining the complex interactions, perturbed dynamics, and exposed
facets and steps by using Molecular Dynamics (MD) to capture the atomic
mechanisms that lead to restructuring.
%This dissertation



%Organized

\section{Metal Systems}
\subsection{Structure}
\subsubsection{Bimetallic Systems}

\section{Adsorbate Interactions on Metal Surfaces}
\subsection{binding sites}
\subsection{patterning}
\subsection{Coverage Dependence}
%Image of particles adsorbed on the surface
%covalent vs charge-interaction vs attractive by not strongly bound
%binding affected by surface facet (electronic, d-band) and generalized coordinate number
% sabatier/volcano plots
%experimental abilities
%dft capabilities
%where molecular dynamics fits
The frontier orbitals on many common small-molecule adsorbates (CO,
NO\textsubscript{x}, O\textsubscript{2}) tend to match up well with the d-band
center of a significant number of relevant metals leading to extremely strong
binding interactions. These binding interactions are often mediated by the
coverage and specific surface facet displayed.

\section{Metal Dynamics}
\subsection{Diffusion}
\subsection{Step-Wandering}


\section{Adsorbate Induced Reconstructions}
% Studies done on clean metal surfaces tend to suffer from pressure and temperature gaps
% high pressure xps, stm, spin echo helium bombardment provide some ways to mimic the environment of industrial catalysts
% Difficult to determine mechanisms because of time and spacial resolution
% DFT good at calculating relative energies for small systems, but the size of these reconstructions makes calculations expensive
% mol dyn. allows us to explore the interactions that lead to adsorbate-induced reconstructions

%\section{Modeling} I think spreading this amongst the earlier sections is better

\subsection{Refaceting}
\subsubsection{Doubling}
\subsection{Island-Formation}


% % uncomment the following lines,
% if using chapter-wise bibliography
%
