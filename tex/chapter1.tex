%Introduction
%  Adsorbate-Metal systems 
%      catalysts
%      devices
%      oxide formation
%    This dissertation
%      structure
%      dynamics
%      reconstruction
%      accurate treatment of electronic interactions and charge transfer effects
%    Organized
%      1: Intro
%      2: Methodology Development
%      3: Pt-CO 557
%      4: Pt/Pd-CO 557
%      5: 112/557/321/765
%      6: MM-FlucQ
%      7: Summary
%
%  Metal Systems:
%      Structure
%        bimetallic (alloys, layers, near surface alloys, etc.)
%      Dynamics
%    Adsorbate Interactions:
%      binding sites
%      patterning
%    Dynamics
%      diffusion
%      step-wandering
%    Reconstruction
%      refaceting
%      doubling
%      island formation

\chapter{INTRODUCTION}
%CHAP1
%Metal surfaces
%  motivation (catalysis, energy generation, biomedical applications)
%  electronic, optical properties (stable at extreme temperatures, pressures for the most part)
%  list off neat things metal surfaces have done (20 or so spread throughout first couple of paragraphs
% Tie into importance of adsorbate interactions
Metal surfaces and nanoparticles play a role in many areas of chemistry,
including catalysis\citep{}, energy generation\citep{}, and biomedical
applications.\citep{} Their displayed facets and morphologies often play a
significant role in their activity\citep{}, selectivity\citep{}, and
stability\citep{}. Bimetallic species and alloys\citep{}, along with supported
surfaces and nanoparticles\citep{} allow for an even greater design space for
various mechanical\citep{}, optical\citep{}, and catalytic properties.\citep{}
In most practical applications the surface of the metal will be exposed to a
variety of atmospheric compositions leading to adsorbed species on the surface
and potentially oxide formation. The interactions between metals and adsorbates
adds another layer of complexity when attempting to fully describe metal
surfaces.

In general, the structure of metal surfaces can be strongly perturbed by the
presence of adsorbates. The extent of adsorbate coverage and the strength of
adsorbate self-interactions can both drastically affect the predicted
low-energy surface. If the preferred low-energy is modified and the kinetic
barrier is not too large, surface reconstructions can result.  Since the
efficacy of metal surfaces strongly depends on the exposed structure, having a
detailed understanding of the mechanism of surface dynamics and restructuring
as caused by adsorbates is of the utmost relevance. This dissertation is
intended to provide a fundamental picture for adsorbate induced reconstruction
on metal surfaces by examining the complex interactions present in these
systems, measuring the perturbed dynamics, and analyzing the effect of
different exposed facets by using Molecular Dynamics (MD) simulations to
capture the atomic mechanisms that lead to restructuring.

This dissertation is organized so that a brief overview of metallic systems,
adsorbate interactions, and adsorbate-induced reconstructions are presented in
Chapter 1. The second chapter focuses on accurately modeling the metal-metal,
metal-adsorbate, and adsorbate-adsorbate interactions, along with introducing
the multiple-minima fluctuating charge (MM-flucQ) method as it relates to the
electrostatic interactions present in the system. Chapter 3 presents work
on the surface reconstructions of Pt (557) and Au (557) surfaces when exposed
to Carbon Monoxide. In Chapter 4 the effect of CO adsorption on a Pt/Pd (557)
subsurface alloy is explored. The fifth chapter more fully examines the effects
of step type and plateau length as they affect the CO-induced restructuring of
Platinum. Moving away from Pt-CO systems, Chapter 6 presents our development of
the MM-flucQ potential and its applications to charge transfer and oxide
formation on Pt surfaces. Finally, Chapter 7 contains a summary of this
work as well as proposals for future directions.

\section{Metals}


\subsection{Structure}
\subsubsection{Bimetallic Systems}
\subsection{Dynamics}
\subsubsection{Diffusion \& Step-Wandering}

\section{Adsorbate Interactions on Metal Surfaces}
\subsection{Binding Sites}
\subsection{Patterning}
\subsection{Coverage Dependence}
%Image of particles adsorbed on the surface
%covalent vs charge-interaction vs attractive by not strongly bound
%binding affected by surface facet (electronic, d-band) and generalized coordinate number
% sabatier/volcano plots
%experimental abilities
%dft capabilities
%where molecular dynamics fits
The frontier orbitals on many common small-molecule adsorbates (CO,
NO\textsubscript{x}, O\textsubscript{2}) tend to match up well with the d-band
center of a significant number of relevant metals leading to extremely strong
binding interactions. These binding interactions are often mediated by the
coverage and specific surface facet displayed.

\section{Metal Dynamics}
\subsection{Diffusion}
\subsection{Step-Wandering}


\section{Adsorbate Induced Reconstructions}
% Studies done on clean metal surfaces tend to suffer from pressure and temperature gaps
% high pressure xps, stm, spin echo helium bombardment provide some ways to mimic the environment of industrial catalysts
% Difficult to determine mechanisms because of time and spacial resolution
% DFT good at calculating relative energies for small systems, but the size of these reconstructions makes calculations expensive
% mol dyn. allows us to explore the interactions that lead to adsorbate-induced reconstructions

%\section{Modeling} I think spreading this amongst the earlier sections is better

\subsection{Refaceting}
\subsubsection{Doubling}
\subsection{Island-Formation}


