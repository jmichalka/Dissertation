
\chapter{STABILITY AND RECONSTRUCTIONS OF PT/PD BIMETALLIC NANOCUBES EXPOSED TO CARBON MONOXIDE}

While metal surfaces are used in various industrial processes, because of their
poor surface area to volume ratio, many methods make use of roughened surfaces
or more preferrably dispersed nanoparticles. The size distribution of the
nanoparticles varies but tends to be on the order of 10s to 100s of nanometers
in size. Additionally, the morphology of these nanoparticles spans from simple
cubes, to octahedra, pyramids, and numerous other morphologies.\citep{} As seen
in previous work\cite{Michalka, Tao}, even fairly stable surfaces can undergo
large scale reconstructions under experimental conditions. This appendix
provides information on the initial setup of bimetallic Pt/Pd nanocubes and the
preliminary results obtained before the project was shelved. 

Nanocubes with edge lengths of 6, 7, and 8 nm were constructed from an ideal Pd
FCC lattice with a lattice constant of 3.89 \AA~  and cut so as to expose the
{100}, {010}, and {001} facets. For each of these three sizes, the outermost 1,
2, or 3 layers were further replaced with Pt to create nine total systems which
are depicted in table \ref{tab:systems}. Work by ?{\it et al.}\citep{} on the
self-distillation of bimetallic nanostructures showed that when the outer shell
of nanoparticles was composed of the metal with a higher melting temperature,
it imposed stability on the confined inner metal. For this reason, Pt was
chosen to compose the outer layers while keeping the core of the nanoparticle
Pd. Additionally, the slightly more favorable Pd-CO interaction may provide a
driving force for reconstructions that will be dependent on the thickness of
the Pt.

\begin{table}
  \caption{PT/PD NANOCUBE COMPOSITION}
  \centering
  \begin{threeparttable}
  \begin{tabular}{ c ccc }
  \hline
  \hline
  \textbf{System} & \textbf{Pd} & \textbf{Pt} &  \textbf{(Pt/total)} \\
  \hline
  6nm-1L & 10976 & 2524  & 0.187 \\
  6nm-2L & 8788  & 4712  & 0.349 \\
  6nm-3L & 6912  & 6588  & 0.488 \\
  7nm-1L & 19652 & 3676  & 0.157 \\
  7nm-2L & 16384 & 6944  & 0.298 \\
  7nm-3L & 13500 & 9828  & 0.421 \\
  8nm-1L & 27436 & 4564  & 0.143 \\
  8nm-2L & 23328 & 8672  & 0.271 \\
  8nm-3L & 19652 & 12348 & 0.386 \\
  \hline
  \hline
  \end{tabular}
  \end{threeparttable}
\label{tab:systems}
\end{table}

Systems were constructed in a non-periodic box and then initially equilibrated
at 300~K in to allow the slight strain of replacing Pd atoms with Pt in the
outer layers to dissipate. Warming over ? ps was performed to bring all systems
up to a simulation temperature of ?~K at which point various amounts of Carbon
Monoxide was introduced into the system, corresponding to , , and ML of
coverage. After another brief period of equilibration during which a
significant amount of the CO adsorbed to the surface, the systems were then run
in the microcanonical (NVE) ensemble for ? ns for data production.



