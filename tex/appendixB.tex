
\chapter{STABILITY AND RECONSTRUCTIONS OF PT/PD BIMETALLIC NANOCUBES}

While metal surfaces are used in various industrial processes, because of their
poor surface area to volume ratio, many methods make use of roughened surfaces
or more preferrably dispersed nanoparticles. The size distribution of the
nanoparticles varies but tends to be on the order of 10s to 100s of nanometers
in size. Additionally, the morphology of these nanoparticles spans from simple
cubes, to octahedra, pyramids, and numerous other morphologies.\citep{} As seen
in previous work\cite{Michalka, Tao}, even fairly stable surfaces can undergo
large scale reconstructions under experimental conditions. This appendix
provides information on the initial setup of bimetallic Pt/Pd nanocubes and the
preliminary results obtained before the project was shelved. 

Nanocubes with edge lengths of 6, 7, and 8 nm were constructed from an ideal Pd
FCC lattice with a lattice constant of 3.89 \AA~  and cut so as to expose the
{100}, {010}, and {001} facets. For each of these three sizes, the outermost 1,
2, or 3 layers were further replaced with Pt to create nine total systems which are 

\begin{table}
  \begin{threeparttable}
  \begin{tabular}{ c cc c }
  \hline
  \hline
  System & Pd & Pt & Composition (Pt/total) \\
  \hline
  6nm_1L & 10976 & 2524  & 18.7\% \\
  6nm_2L & 8788  & 4712  & 34.9\% \\
  6nm_3L & 6912  & 6588  & \\
  7nm_1L & 19652 & 3676  & \\
  7nm_2L & 16384 & 6944  & \\
  7nm_3L & 13500 & 9828  & \\
  8nm_1L & 27436 & 4564  & \\
  8nm_2L & 23328 & 8672  & \\
  8nm_3L & 19652 & 12348 & \\
  \hline
  \hline
  \end{tabular}
  \begin{tablenotes}

  \end{tablenotes}
  \end{threeparttable}
\label{tab:systems}
\end{tabular}
